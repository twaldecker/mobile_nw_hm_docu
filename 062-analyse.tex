\utsection{Analyse}{Thomas Waldecker}\label{sec:analyse}

\subsection{Vorbereitung}
\subsubsection{Tracen auf dem Um Interface mit einem Nokia 3310}
Das Um Interface kann mit einem Nokia 3310 Mobiltelefon mit einem Adapter der zwischen den Akku und dem Telefon gesteckt wird und auf 4 Kontakte des FBus??.

\begin{lstlisting}[label=config:gammu,caption={Konfigurationsdatei für gammu und dem verwendeten Adapter}]
[gammu]

port = /dev/ttyUSB0
model = 6110
connection = mbus
synchronizetime = yes
logfile = 
logformat = nothing
use_locking = yes
gammuloc = 
\end{lstlisting}

\begin{lstlisting}[label=command:gammu,caption={Aufruf von Gammu}]
sudo gammu --nokiadebug nhm5_587.txt v20-25,v18-19
Debug Trace Mode -- wumpus 2003
Loading
Activating ranges:
  20-25 verbose=1
  18-19 verbose=1
Debug Trace Enabled
Press Ctrl+C to interrupt...
<1805> MDI:m2d/FROM_MCU_TO_FBUS
t=0002 nr=0f: D 05: 1e 0c 00 40 00 06 01 01 70 01 01 47 
<198E> MDI:d2m/FROM_FBUS_TO_MCU
t=0003 nr=10: D 8E: 1e 00 0c 7f 00 02 40 07 
\end{lstlisting}

\subsection{Intra BSC Handover mit OpenBSC und zwei nanoBTS}


\subsubsection{Aufbau}
In einem Raum (Mobile Netze Labor) wurden an zwei Ecken jeweils eine nanoBTS aufgestellt und mit dem Laborrechner verbunden. 
