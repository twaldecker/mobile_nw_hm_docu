\utsection{Zusammenfassung und Ausblick}{Thomas Waldecker}

Die Aufgabenstellung war in der gegebenen Zeit mit drei Personen nicht zu bewältigen. Dennoch wurde das Projekt bis zu einem akzeptablen Punkt fertiggestellt an dem die Grundlagen funktionieren und worauf eine andere Gruppe aufbauen kann.

Die Entscheidung zuerst das System mit OpenBSC und den nanoBTS zuerst in Betrieb zu nehmen war richtig und wichtig um die Grundlagen von GSM Basisstationen zu verstehen und einen funktionierenden Handover analysieren zu können.

Die zugehörige Vorlesung "`Mobile Netze"' hat einen großen umfassenden Überblick über die verschiedenen drahtlosen Funktechnologien der Telekommunikation gegeben. Durch das Untersuchen von OpenBSC und Erweitern von OpenBTS in der Projektarbeit bekamen wir sehr tiefe Einblicke in die Arbeitsweise und den offenen Quelltext von funktionierenden GSM Basisstationen.

Als nächsten Schritte in der Implementierung der Handoverfunktionalität zwischen zwei OpenBTS wäre eine Abis Kommunikationsschnittstelle und die geforderten Funktionalitäten bereitzustellen und das Weiterleiten des Traffic Channels an den neuen Zeitschlitz / an die neue BTS. Unabhängig davon muss auch die Logik der Entscheidung für einen Handover entwickelt werden.
