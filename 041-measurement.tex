\section{Measurement Report}
Measurement Reports enthalten Messwerte bzgl. der Empfangsleistung, Empfangsqualität und Informationen zu Nachbarzellen. Sie werden beim Einbuchen in das Netzwerk und während eines Gesprächs (ca. 2 mal in der Sekunde) von der MS an die BTS gesandt. Measurement Reports sind im RR-Sublayer (Radio Resource) angesiedelt und mit dem Nachrichtentyp \textit{MEASUREMENT REPORT} gekennzeichnet. Die Messwerte sind für das Weiterreichen (Handover) der MS von großer Bedeutung.

OpenBTS verwaltet diese Messwerte intern in einer eigenen Klasse, bietet aber auch die Möglichkeit, diese in eine externe SQL-DB abzulegen. Mit der Option \verb|Control.Reporting.PhysStatusTable| kann der Pfad der Datenbank angegeben werden:
\begin{verbatim}
OpenBTS> config Control.Reporting.PhysStatusTable \
/etc/OpenBTS/phystatus.db
\end{verbatim}

Leider werden keinerlei Informationen bzgl. der Nachbarzellen in die Datenbank eingetragen. Deshalb musste die entsprechende Tabelle der Datenbank um weitere Felder für die Nachbarzellen erweitert und die Methode \verb|PhysicalStatus::setPhysical()| angepasst werden.
Zusätzlich wurde ein neuer CLI-Befehl namens \verb|showmr| implementiert, welcher den Inhalt der Measurement Report DB entsprechend formatiert und auflistet.

\newpage
Beispielausgabe von \verb|showmr|:
\begin{figure}[ht]
\setbox0\vbox{\small
\begin{verbatim}
OpenBTS> showmr
############################################################
                Measurement Report:
############################################################
CN_TN_TYPE_OFFSET               =       C0T0 SDCCH/4-0
ARFCN                           =       867
ACCESSED                        =       1330702677
RSSI                            =       -63.750000
TIME_ERR                        =       -0.222656
TIME_ADVC                       =       1
TRANS_PWR                       =       30 dBm
FER                             =       0.000000
RXLEV_FULL_SERVING_CELL         =       -48 dBm
RXLEV_SUB_SERVING_CELL          =       -48 dBm
RXQUAL_FULL_SERVING_CELL_BER    =       0.181000 dBm
RXQUAL_SUB_SERVING_CELL_BER     =       0.181000 dBm
NO_NCELL                        =       1
RXLEV_CELL_1 = 0, BCCH_FREQ_CELL_1 = 0, BSIC_CELL_1 = 0
RXLEV_CELL_2 = 0, BCCH_FREQ_CELL_2 = 0, BSIC_CELL_2 = 0
RXLEV_CELL_3 = 0, BCCH_FREQ_CELL_3 = 0, BSIC_CELL_3 = 0
RXLEV_CELL_4 = 0, BCCH_FREQ_CELL_4 = 0, BSIC_CELL_4 = 0
RXLEV_CELL_5 = 0, BCCH_FREQ_CELL_5 = 0, BSIC_CELL_5 = 0
RXLEV_CELL_6 = -33, BCCH_FREQ_CELL_6 = 63, BSIC_CELL_6 = 1
############################################################
CN_TN_TYPE_OFFSET               =       C0T1 TCH/F
ARFCN                           =       867
ACCESSED                        =       1330696371
RSSI                            =       -57.250000
TIME_ERR                        =       0.263672
TIME_ADVC                       =       1
TRANS_PWR                       =       30 dBm
FER                             =       0.042869
RXLEV_FULL_SERVING_CELL         =       -48 dBm
RXLEV_SUB_SERVING_CELL          =       -48 dBm
RXQUAL_FULL_SERVING_CELL_BER    =       0.000000 dBm
RXQUAL_SUB_SERVING_CELL_BER     =       0.000000 dBm
NO_NCELL                        =       7
RXLEV_CELL_1 = 0, BCCH_FREQ_CELL_1 = 0, BSIC_CELL_1 = 0
RXLEV_CELL_2 = 0, BCCH_FREQ_CELL_2 = 0, BSIC_CELL_2 = 0
RXLEV_CELL_3 = 0, BCCH_FREQ_CELL_3 = 0, BSIC_CELL_3 = 0
RXLEV_CELL_4 = 0, BCCH_FREQ_CELL_4 = 0, BSIC_CELL_4 = 0
RXLEV_CELL_5 = 0, BCCH_FREQ_CELL_5 = 0, BSIC_CELL_5 = 0
RXLEV_CELL_6 = 0, BCCH_FREQ_CELL_6 = 0, BSIC_CELL_6 = 0

############################################################
\end{verbatim}
}
\centerline{\fbox{\box0}}
\end{figure}
\newpage

\underline{Erläuterungen zur Beispielausgabe}

Der erste Measurement Report wurde um \verb|1330702677| (Unix-Time, entspricht 02.03.2012 - 16:37:57 Realzeit) im \verb|SDCCH| (\textit{Standalone Dedicated Control Channel}) mit der Nummer 0 (von 4 Möglichen) auf der \verb|ARFCN 867| gesendet. Die empfangene Signalstärke (\verb|RSSI = Received Signal Strength Indication|) betrug \verb|-63.75 dBm|. Der zugeordnete Timing Advance Parameter der MS betrug 1 Symbolperiode und wies einen Fehler (\verb|TIME_ERR|), d.h. Zeitversatz von \verb|-0.222656| Symbolperioden auf. Die Sendeleistung der MS betug \verb|30 dBm| und hatte bis dato eine \verb|Uplink-FER| (= Frame Erasure Rate; gibt das Verhältnis zwischen gelöschten Frames und der Gesamtanzahl der Frames an) von \verb|0|. Der Empfangspegel der verwendeten Zelle (\verb|RXLEV_FULL_SERVING_CELL|) betrug \verb|-48 dBm| und die Empfangsqualität \verb|RXQUAL_FULL_SERVING_CELL_BER = 0.181000 dBm|. Die Angaben \verb|SUB| und \verb|FULL| bei der Empfangsleistung und -qualität beziehen sich auf die Verwendung von DTX (Discontinuous Transmission). \verb|FULL| bezieht dabei alle Frames mit ein, also auch die zu dessen Zeit keine Sprache gesendet wurde. \verb|SUB| hingegen nur die effektiven "`Sprachframes"'. Da jeweils beide Werte im obigen Beispiel gleich sind, kann man davon ausgehen das kein DTX verwendet wurde. Die restlichen Angaben beziehen sich auf die Nachbarzellen. \verb|NO_NCELL| gibt die Anzahl der sichbaren Nachbarzellen an. Dabei gibt es zwei Sonderfälle: \verb|NO_NCELL=0| - es existieren keine Messwerte, \verb|NO_NCELL=7| - es existieren keine Nachbarzellen. Im obigen Beispiel sieht die MS 1 Nachbarzelle (Nr. 6) mit einer Empfangsleistung von \verb|-33 dBm|. 